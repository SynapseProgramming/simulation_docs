\documentclass[11pt]{article}
\usepackage[T1]{fontenc}
\usepackage{listings}
\usepackage{hyperref}
\usepackage{color}

\definecolor{dkgreen}{rgb}{0,0.6,0}
\definecolor{gray}{rgb}{0.5,0.5,0.5}
\definecolor{mauve}{rgb}{0.58,0,0.82}

\lstset{frame=tb,
  language=C++,
  aboveskip=3mm,
  belowskip=3mm,
  showstringspaces=false,
  columns=flexible,
  basicstyle={\small\ttfamily},
  numbers=none,
  numberstyle=\tiny\color{gray},
  keywordstyle=\color{blue},
  commentstyle=\color{dkgreen},
  stringstyle=\color{mauve},
  breaklines=true,
  breakatwhitespace=true,
  tabsize=3
}

\setlength{\parindent}{0pt}
\setlength{\parskip}{0pt plus 0.5ex}
%% adjust spacing for all itemize/enumerate

\begin{document}

% TODO: add in more links whenever possible
\section{Useful Resources}
Robot modelling in urdf is rather tedious. Here are a couple of guides which I have found useful.

\begin{itemize}
 \item {
       \href{https://nu-msr.github.io/me495_site/lecture06_modeling.html}{URDF/xacro guide}
       }
\end{itemize}


\section{Installation}
\begin{enumerate}
 \item{Install the joint state publisher:
       \begin{lstlisting}[language=bash]
               $ sudo apt install ros-melodic-joint-state-publisher-gui
       \end{lstlisting}
       }
 \item{
       Next, git clone realsense package into your catkin folder and run catkinmake.
       
       \begin{lstlisting}[language=bash]
       $ git clone https://github.com/pal-robotics/realsense_gazebo_plugin.git
       \end{lstlisting}
       }
\end{enumerate}

\section{Package Overview}
\subsection{urdf}
The urdf folder is used to store (.urdf.xacro) and (.gazebo.xacro) files.
\begin{itemize}
 \item {
       (.urdf.xacro) files describe the transformations between the various links of the robot, along with other variables such as intertia.
       }
 \item{
       (.gazebo.xacro) files would describe the kinematic properties of the robot such as velocity limits, coefficient of friction etc.
       }
\end{itemize}
\subsubsection{Intel Realsense D435 camera}
\begin{itemize}
 \item {
       The (\_d435.gazebo.xacro) file is used to describe the various kinematic properties of the depth camera.
       }
 \item{
       The (\_d435.urdf.xacro) file is used to describe the various frames of the depth camera.
       
       The (<camera>\_bottom\_screw\_frame) is used as the main reference point for the depth camera. It is centered about the bottom tripod mount of the depth camera.
       Thus, to define the position of the depth camera on the robot, we would have to provide the transformation from base link to the bottom screw frame link
       }
\end{itemize}

\subsection{meshes}
The meshes folder is used to store the various mesh files of the robot(.stl .dae).
The mesh files are used to visualize various components of the robot(sensors, chassis) and to enable accurate collisions within gazebo.

\subsection{rviz}
The rviz folder is used to store the various rviz configs.

% TODO: update launch section when more launch files are created
\subsection{launch}
\begin{itemize}
 \item{
       
       The launch folder is used to store roslaunch files. Launch files titled simulate\_<robotname>.launch are used to launch the gazebo simulation environment and rviz.
       }
 \item{
       Launch files titled view\_<robot>.launch are used to view the robots urdf model in rviz.
       }
\end{itemize}


% TODO: complete the section on how to scale and manipulate stl files so that the mesh could be easily integrated into rviz

\section{Mesh Manipulation}
test test 
\end{document}
